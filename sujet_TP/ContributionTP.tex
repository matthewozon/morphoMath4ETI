\documentclass[12pt]{article}

\usepackage[latin1]{inputenc}
\usepackage[french]{babel}
\usepackage{indentfirst}
\usepackage{amssymb}
%\usepackage{makeidx}
%\usepackage[tcidvi]{graphicx}
\usepackage{graphicx}
\usepackage{amsmath}
\usepackage{amsfonts}
\usepackage{lineno}
\usepackage{float}
\oddsidemargin  0.0in
	\evensidemargin 0.0in
	\textwidth      7.0in
	\headheight     0.0in
	\topmargin      0.0in
	\textheight=7.5in

\begin{document}


\begin{center}
\Large{3) Reconstruction par marqueur.}
\end{center}


\bigskip

\normalsize

\begin{figure}[H]
\begin{center}
\includegraphics[]{ReconstructionParMarqueurs.jpg}
\caption{\textit{Image d'un tapis d'�pines de pin, seuill�e pour en faire une image binaire. On construit une image ''marqueur'' constitu�e de 10 ''points'' indiquant des zones � reconstruire. On reconstruit une nouvelle image constitu�e des zones blanches de l'image seuill�e qui avaient au moins un pixel commun avec un pixel blanc de l'image marqueur.}}
\label{rpm}
\end{center}
\end{figure}%

\newpage

Pour comprendre la reconstruction par marqueurs, tapez le programme Matlab ci-dessous dont une trace d'ex�cution est donn�e par la Fig. \ref{rpm}. Comprenez-en les diff�rentes �tapes et commentez-les.

\begin{verbatim}
clear all;close all;clc;
set(gcf,'color','w');%fond blanc
f=imread('aiguilles_pin.jpg');%� remplacer par une image de votre choix
f=f(1:512,1:512);
[a,b]=size(f);
subplot(2,2,1);imshow(f);title('image originale');
f=(f>200);
whos f
f=im2double(f);
subplot(2,2,2);imshow(f);title('image seuill�e');
marq=zeros(a,b);% �quivalent � : m=zeros(size(f));
for k=1:10;
    marq(ceil(a*rand),ceil(b*rand))=1;
end;
marq=imdilate(marq,ones(20),'same');
subplot(2,2,3);imshow(marq);title('marqueur');
g=imreconstruct(marq,f);
subplot(2,2,4);imshow(g);title('image reconstruite');
\end{verbatim}

\bigskip

Travail personnel : Comment pourrait-on se servir de la reconstruction par marqueur pour �liminer les aiguilles qui touchent le bord de l'image (indication : prendre comme marqueur le bord de l'image)

\end{document}


